% !Mode:: "TeX:UTF-8" 






\BiSection{3.20}{Figures}

\fancyhead[R]{本题3.20由QC.Z完成}



解:

\scalebox{3}{(a)}

(a)重画于图1

		\begin{figure}[H] %H为当前位置,!htb为忽略美学标准,htbp为浮动图形
	\begin{minipage}{\linewidth}
		\includegraphics[width=1\linewidth]{3.20-1}
	\end{minipage}
	\caption*{图1} %最终文档中希望显示的图片标题
\end{figure}




在$V_{out}$点用基尔霍夫电流定律Kirchhoff’s Current Law(KCL)

		\begin{figure}[H] %H为当前位置,!htb为忽略美学标准,htbp为浮动图形
	\begin{minipage}{\linewidth}
		\includegraphics{3.20-2}
	\end{minipage}
\end{figure}

		\begin{figure}[H] %H为当前位置,!htb为忽略美学标准,htbp为浮动图形
	\begin{minipage}{\linewidth}
		\includegraphics{3.20-3}
	\end{minipage}
\end{figure}

		\begin{figure}[H] %H为当前位置,!htb为忽略美学标准,htbp为浮动图形
	\begin{minipage}{\linewidth}
		\includegraphics{3.20-4}
	\end{minipage}
\end{figure}

		\begin{figure}[H] %H为当前位置,!htb为忽略美学标准,htbp为浮动图形
	\begin{minipage}{\linewidth}
		\includegraphics{3.20-5}
	\end{minipage}
\end{figure}

		\begin{figure}[H] %H为当前位置,!htb为忽略美学标准,htbp为浮动图形
	\begin{minipage}{\linewidth}
		\includegraphics{3.20-6}
	\end{minipage}
\end{figure}

\scalebox{3}{(b)}

(b)重画于图2


		\begin{figure}[H] %H为当前位置,!htb为忽略美学标准,htbp为浮动图形
	\begin{minipage}{\linewidth}
		\includegraphics[width=1\linewidth]{3.20-7}
	\end{minipage}
	\caption*{图2} %最终文档中希望显示的图片标题
\end{figure}



在$V_{out}$点用基尔霍夫电流定律Kirchhoff’s Current Law(KCL)


		\begin{figure}[H] %H为当前位置,!htb为忽略美学标准,htbp为浮动图形
	\begin{minipage}{\linewidth}
		\includegraphics{3.20-8}
	\end{minipage}
\end{figure}

\begin{figure}[H] %H为当前位置,!htb为忽略美学标准,htbp为浮动图形
	\begin{minipage}{\linewidth}
		\includegraphics{3.20-9}
	\end{minipage}
\end{figure}

\begin{figure}[H] %H为当前位置,!htb为忽略美学标准,htbp为浮动图形
	\begin{minipage}{\linewidth}
		\includegraphics{3.20-10}
	\end{minipage}
\end{figure}

\begin{figure}[H] %H为当前位置,!htb为忽略美学标准,htbp为浮动图形
	\begin{minipage}{\linewidth}
		\includegraphics{3.20-11}
	\end{minipage}
\end{figure}















\scalebox{3}{(c)}

(c)重画于图3

		\begin{figure}[H] %H为当前位置,!htb为忽略美学标准,htbp为浮动图形
	\begin{minipage}{\linewidth}
		\includegraphics[width=1\linewidth]{3.20-12}
	\end{minipage}
	\caption*{图3} %最终文档中希望显示的图片标题
\end{figure}




在$V_{out}$点用基尔霍夫电流定律Kirchhoff’s Current Law(KCL)

		\begin{figure}[H] %H为当前位置,!htb为忽略美学标准,htbp为浮动图形
	\begin{minipage}{\linewidth}
		\includegraphics{3.20-13}
	\end{minipage}
\end{figure}

\begin{figure}[H] %H为当前位置,!htb为忽略美学标准,htbp为浮动图形
	\begin{minipage}{\linewidth}
		\includegraphics{3.20-14}
	\end{minipage}
\end{figure}

\begin{figure}[H] %H为当前位置,!htb为忽略美学标准,htbp为浮动图形
	\begin{minipage}{\linewidth}
		\includegraphics{3.20-15}
	\end{minipage}
\end{figure}

\begin{figure}[H] %H为当前位置,!htb为忽略美学标准,htbp为浮动图形
	\begin{minipage}{\linewidth}
		\includegraphics{3.20-16}
	\end{minipage}
\end{figure}















\scalebox{3}{(d)}

(d)重画于图4

		\begin{figure}[H] %H为当前位置,!htb为忽略美学标准,htbp为浮动图形
	\begin{minipage}{\linewidth}
		\includegraphics[width=1\linewidth]{3.20-17}
	\end{minipage}
	\caption*{图4} %最终文档中希望显示的图片标题
\end{figure}




在$V_{X}$点和$V_{out}$点用基尔霍夫电流定律Kirchhoff’s Current Law(KCL)

		\begin{figure}[H] %H为当前位置,!htb为忽略美学标准,htbp为浮动图形
	\begin{minipage}{\linewidth}
		\includegraphics{3.20-18}
	\end{minipage}
\end{figure}

\begin{figure}[H] %H为当前位置,!htb为忽略美学标准,htbp为浮动图形
	\begin{minipage}{\linewidth}
		\includegraphics{3.20-19}
	\end{minipage}
\end{figure}

\begin{figure}[H] %H为当前位置,!htb为忽略美学标准,htbp为浮动图形
	\begin{minipage}{\linewidth}
		\includegraphics{3.20-20}
	\end{minipage}
\end{figure}

\begin{figure}[H] %H为当前位置,!htb为忽略美学标准,htbp为浮动图形
	\begin{minipage}{\linewidth}
		\includegraphics{3.20-21}
	\end{minipage}
\end{figure}

\begin{figure}[H] %H为当前位置,!htb为忽略美学标准,htbp为浮动图形
	\begin{minipage}{\linewidth}
		\includegraphics{3.20-22}
	\end{minipage}
\end{figure}

\begin{figure}[H] %H为当前位置,!htb为忽略美学标准,htbp为浮动图形
	\begin{minipage}{\linewidth}
		\includegraphics[width=1\linewidth]{3.20-23}
	\end{minipage}
\end{figure}

\begin{figure}[H] %H为当前位置,!htb为忽略美学标准,htbp为浮动图形
	\begin{minipage}{\linewidth}
		\includegraphics[width=1\linewidth]{3.20-24}
	\end{minipage}
\end{figure}

\begin{figure}[H] %H为当前位置,!htb为忽略美学标准,htbp为浮动图形
	\begin{minipage}{\linewidth}
		\includegraphics{3.20-25}
	\end{minipage}
\end{figure}

\begin{figure}[H] %H为当前位置,!htb为忽略美学标准,htbp为浮动图形
	\begin{minipage}{\linewidth}
		\includegraphics{3.20-26}
	\end{minipage}
\end{figure}













\scalebox{3}{(e)}

(e)重画于图5

		\begin{figure}[H] %H为当前位置,!htb为忽略美学标准,htbp为浮动图形
	\begin{minipage}{\linewidth}
		\includegraphics[width=1\linewidth]{3.20-27}
	\end{minipage}
	\caption*{图5} %最终文档中希望显示的图片标题
\end{figure}




在$V_{X}$点和$V_{out}$点用基尔霍夫电流定律Kirchhoff’s Current Law(KCL)


		\begin{figure}[H] %H为当前位置,!htb为忽略美学标准,htbp为浮动图形
	\begin{minipage}{\linewidth}
		\includegraphics{3.20-28}
	\end{minipage}
\end{figure}

\begin{figure}[H] %H为当前位置,!htb为忽略美学标准,htbp为浮动图形
	\begin{minipage}{\linewidth}
		\includegraphics{3.20-29}
	\end{minipage}
\end{figure}

\begin{figure}[H] %H为当前位置,!htb为忽略美学标准,htbp为浮动图形
	\begin{minipage}{\linewidth}
		\includegraphics{3.20-30}
	\end{minipage}
\end{figure}

\begin{figure}[H] %H为当前位置,!htb为忽略美学标准,htbp为浮动图形
	\begin{minipage}{\linewidth}
		\includegraphics{3.20-31}
	\end{minipage}
\end{figure}

\begin{figure}[H] %H为当前位置,!htb为忽略美学标准,htbp为浮动图形
	\begin{minipage}{\linewidth}
		\includegraphics[width=1\linewidth]{3.20-32}
	\end{minipage}
\end{figure}

\begin{figure}[H] %H为当前位置,!htb为忽略美学标准,htbp为浮动图形
	\begin{minipage}{\linewidth}
		\includegraphics[width=1\linewidth]{3.20-33}
	\end{minipage}
\end{figure}

\begin{figure}[H] %H为当前位置,!htb为忽略美学标准,htbp为浮动图形
	\begin{minipage}{\linewidth}
		\includegraphics{3.20-34}
	\end{minipage}
\end{figure}

\begin{figure}[H] %H为当前位置,!htb为忽略美学标准,htbp为浮动图形
	\begin{minipage}{\linewidth}
		\includegraphics{3.20-35}
	\end{minipage}
\end{figure}










